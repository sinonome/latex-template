\usepackage{listings}
\lstset{%
    language=Python,
    basicstyle={\ttfamily},
    breaklines=true,
    columns=[l]{fullflexible},
    lineskip=-0.5zw,
}
% \newtheoremstyle{mystyle}%      % スタイル名
%     {}%                         % 上部スペース
%     {}%                         % 下部スペース
%     {\normalfont}%              % 本文フォント
%     {}%                         % インデント量
%     {\bf}%                      % 見出しフォント
%     {}%                         % 見出し後の句読点, '.'
%     {\newline}%                        % 見出し後のスペース, ' ' or \newline
%     {\underline{\thmname{#1}\thmnumber{#2.}\thmnote{ - #3}}}%
%                                 % 見出しの書式 (can be left empty, meaning `normal')
% \theoremstyle{mystyle}      % スタイルの適用

\newtheoremstyle{examplestyle}% name of the style to be used
   {10mm}% measure of space to leave above the theorem. E.g.: 3pt
   {10mm}% measure of space to leave below the theorem. E.g.: 3pt
   {\slshape}% name of font to use in the body of the theorem
   {6pt}% measure of space to indent
   {\bfseries}% name of head font
   {}% punctuation between head and body
   {\newline}% space after theorem head
   {\underline{\thmname{#1}\thmnumber{#2.}\thmnote{ - #3}}}% Manually specify head
\theoremstyle{examplestyle}
